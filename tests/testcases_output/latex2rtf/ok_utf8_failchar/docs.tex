\documentclass{report}
\usepackage{hyperref}
% WARNING: THIS SHOULD BE MODIFIED DEPENDING ON THE LETTER/A4 SIZE
\oddsidemargin 0cm
\evensidemargin 0cm
\marginparsep 0cm
\marginparwidth 0cm
\parindent 0cm
\textwidth 16.5cm

\ifpdf
  \usepackage[pdftex]{graphicx}
\else
  \usepackage[dvips]{graphicx}
\fi

\begin{document}
% special variable used for calculating some widths.
\newlength{\tmplength}
\chapter{Unit ok{\_}utf8{\_}failchar}
\section{Description}
UTF{-}8 file with BOM ({\$}EF, {\$}BB, {\$}BF).

This comment contains three code points that fail to translate to UCS2, they are in the Vietnamese (nôm) translation near the bottom of this page and should not trigger an error with Unicode Delphi but being replaced by the default fail char "?".

Text below is borrowed from the webpage 

I Can Eat Glass\\{}

Sanskrit: काचं शक्नोम्यत्तुम् । नोपहिनस्ति माम् ॥\\{} Sanskrit (standard transcription): kācaṃ śaknomyattum; nopahinasti mām.\\{} Classical Greek: ὕαλον ϕαγεῖν δύναμαι· τοῦτο οὔ με βλάπτει.\\{} Greek (monotonic): Μπορώ να φάω σπασμένα γυαλιά χωρίς να πάθω τίποτα.\\{} Greek (polytonic): Μπορῶ νὰ φάω σπασμένα γυαλιὰ χωρὶς νὰ πάθω τίποτα.\\{} Etruscan: (NEEDED)\\{} Latin: Vitrum edere possum; mihi non nocet.\\{} Old French: Je puis mangier del voirre. Ne me nuit.\\{} French: Je peux manger du verre, ça ne me fait pas mal.\\{} Provençal / Occitan: Pòdi manjar de veire, me nafrariá pas.\\{} Québécois: J'peux manger d'la vitre, ça m'fa pas mal.\\{} Walloon: Dji pou magnî do vêre, çoula m' freut nén må.\\{} Champenois: (NEEDED)\\{} Lorrain: (NEEDED)\\{} Picard: Ch'peux mingi du verre, cha m'foé mie n'ma.\\{} Corsican/Corsu: (NEEDED)\\{} Jèrriais: (NEEDED)\\{} Kreyòl Ayisyen (Haitï): Mwen kap manje vè, li pa blese'm.\\{} Basque: Kristala jan dezaket, ez dit minik ematen.\\{} Catalan / Català: Puc menjar vidre, que no em fa mal.\\{} Spanish: Puedo comer vidrio, no me hace daño.\\{} Aragones: Puedo minchar beire, no me'n fa mal .\\{} Mallorquín: (NEEDED)\\{} Galician: Eu podo xantar cristais e non cortarme.\\{} European Portuguese: Posso comer vidro, não me faz mal.\\{} Brazilian Portuguese (8): Posso comer vidro, não me machuca.\\{} Caboverdiano/Kabuverdianu (Cape Verde): M' podê cumê vidru, ca ta maguâ{-}m'.\\{} Papiamentu: Ami por kome glas anto e no ta hasimi daño.\\{} Italian: Posso mangiare il vetro e non mi fa male.\\{} Milanese: Sôn bôn de magnà el véder, el me fa minga mal.\\{} Roman: Me posso magna' er vetro, e nun me fa male.\\{} Napoletano: M' pozz magna' o'vetr, e nun m' fa mal.\\{} Venetian: Mi posso magnare el vetro, no'l me fa mae.\\{} Zeneise (Genovese): Pòsso mangiâ o veddro e o no me fà mâ.\\{} Sicilian: Puotsu mangiari u vitru, nun mi fa mali.\\{} Campinadese (Sardinia): (NEEDED)\\{} Lugudorese (Sardinia): (NEEDED)\\{} Romansch (Grischun): Jau sai mangiar vaider, senza che quai fa donn a mai.\\{} Romany / Tsigane: (NEEDED)\\{} Romanian: Pot să mănânc sticlă și ea nu mă rănește.\\{} Esperanto: Mi povas manĝi vitron, ĝi ne damaĝas min.\\{} Pictish: (NEEDED)\\{} Breton: (NEEDED)\\{} Cornish: Mý a yl dybry gwéder hag éf ny wra ow ankenya.\\{} Welsh: Dw i'n gallu bwyta gwydr, 'dyw e ddim yn gwneud dolur i mi.\\{} Manx Gaelic: Foddym gee glonney agh cha jean eh gortaghey mee.\\{} Old Irish (Latin): Con·iccim ithi nglano. Ním·géna.\\{} Irish: Is féidir liom gloinne a ithe. Ní dhéanann sí dochar ar bith dom.\\{} Ulster Gaelic: Ithim{-}sa gloine agus ní miste damh é.\\{} Scottish Gaelic: S urrainn dhomh gloinne ithe; cha ghoirtich i mi.\\{} Old Norse (Latin): Ek get etið gler án þess að verða sár.\\{} Norsk / Norwegian (Nynorsk): Eg kan eta glas utan å skada meg.\\{} Norsk / Norwegian (Bokmål): Jeg kan spise glass uten å skade meg.\\{} Føroyskt / Faroese: Eg kann eta glas, skaðaleysur.\\{} Íslenska / Icelandic: Ég get etið gler án þess að meiða mig.\\{} Svenska / Swedish: Jag kan äta glas utan att skada mig.\\{} Dansk / Danish: Jeg kan spise glas, det gør ikke ondt på mig.\\{} Sønderjysk: Æ ka æe glass uhen at det go mæ naue.\\{} Frysk / Frisian: Ik kin glês ite, it docht me net sear.\\{} Nederlands / Dutch: Ik kan glas eten, het doet mij geen kwaad.\\{} Kirchröadsj/Bôchesserplat: Iech ken glaas èèse, mer 't deet miech jing pieng.\\{} Afrikaans: Ek kan glas eet, maar dit doen my nie skade nie.\\{} Lëtzebuergescht / Luxemburgish: Ech kan Glas iessen, daat deet mir nët wei.\\{} Deutsch / German: Ich kann Glas essen, ohne mir zu schaden.\\{} Ruhrdeutsch: Ich kann Glas verkasematuckeln, ohne dattet mich wat jucken tut.\\{} Langenfelder Platt: Isch kann Jlaas kimmeln, uuhne datt mich datt weh dääd.\\{} Lausitzer Mundart ("Lusatian"): Ich koann Gloos assn und doas dudd merr ni wii.\\{} Odenwälderisch: Iech konn glaasch voschbachteln ohne dass es mir ebbs daun doun dud.\\{} Sächsisch / Saxon: 'sch kann Glos essn, ohne dass'sch mer wehtue.\\{} Pfälzisch: Isch konn Glass fresse ohne dasses mer ebbes ausmache dud.\\{} Schwäbisch / Swabian: I kå Glas frässa, ond des macht mr nix!\\{} Bayrisch / Bavarian: I koh Glos esa, und es duard ma ned wei.\\{} Allemannisch: I kaun Gloos essen, es tuat ma ned weh.\\{} Schwyzerdütsch (Zürich): Ich chan Glaas ässe, das schadt mir nöd.\\{} Schwyzerdütsch (Luzern): Ech cha Glâs ässe, das schadt mer ned.\\{} Plautdietsch: (NEEDED)\\{} Hungarian: Meg tudom enni az üveget, nem lesz tőle bajom.\\{} Suomi / Finnish: Voin syödä lasia, se ei vahingoita minua.\\{} Sami (Northern): Sáhtán borrat lása, dat ii leat bávččas.\\{} Erzian: Мон ярсан суликадо, ды зыян эйстэнзэ а ули.\\{} Northern Karelian: Mie voin syvvä lasie ta minla ei ole kipie.\\{} Southern Karelian: Minä voin syvvä st'oklua dai minule ei ole kibie.\\{} Vepsian: (NEEDED)\\{} Votian: (NEEDED)\\{} Livonian: (NEEDED)\\{} Estonian: Ma võin klaasi süüa, see ei tee mulle midagi.\\{} Latvian: Es varu ēst stiklu, tas man nekaitē.\\{} Lithuanian: Aš galiu valgyti stiklą ir jis manęs nežeidžia\\{} Old Prussian: (NEEDED)\\{} Sorbian (Wendish): (NEEDED)\\{} Czech: Mohu jíst sklo, neublíží mi.\\{} Slovak: Môžem jesť sklo. Nezraní ma. Polska / Polish: Mogę jeść szkło i mi nie szkodzi.\\{} Slovenian: Lahko jem steklo, ne da bi mi škodovalo.\\{} Croatian: Ja mogu jesti staklo i ne boli me.\\{} Serbian (Latin): Ja mogu da jedem staklo.\\{} Serbian (Cyrillic): Ја могу да једем стакло.\\{} Macedonian: Можам да јадам стакло, а не ме штета.\\{} Russian: Я могу есть стекло, оно мне не вредит.\\{} Belarusian (Cyrillic): Я магу есці шкло, яно мне не шкодзіць.\\{} Belarusian (Lacinka): Ja mahu jeści škło, jano mne ne škodzić.\\{} Ukrainian: Я можу їсти скло, і воно мені не зашкодить.\\{} Bulgarian: Мога да ям стъкло, то не ми вреди.\\{} Georgian: მინას ვჭამ და არა მტკივა.\\{} Armenian: Կրնամ ապակի ուտել և ինծի անհանգիստ չըներ։\\{} Albanian: Unë mund të ha qelq dhe nuk më gjen gjë.\\{} Turkish: Cam yiyebilirim, bana zararı dokunmaz.\\{} Turkish (Ottoman): جام ييه بلورم بڭا ضررى طوقونمز\\{} Bangla / Bengali: আমি কাঁচ খেতে পারি, তাতে আমার কোনো ক্ষতি হয় না।\\{} Marathi: मी काच खाऊ शकतो, मला ते दुखत नाही.\\{} Kannada (ಕನ್ನಡ): ಎಲ್ಲಾದರೂ ಇರು, ಎಂತಾದರು ಇರು, ಎಂದೆಂದಿಗೂ ನೀ ಕನ್ನಡವಾಗಿರು, ಕನ್ನಡವೇ ಸತ್ಯ.. ಕನ್ನಡವೇ ನಿತ್ಯ..\\{} Hindi: मैं काँच खा सकता हूँ और मुझे उससे कोई चोट नहीं पहुंचती.\\{} Tamil: நான் கண்ணாடி சாப்பிடுவேன், அதனால் எனக்கு ஒரு கேடும் வராது.\\{} Telugu: నేను గాజు తినగలను మరియు అలా చేసినా నాకు ఏమి ఇబ్బంది లేదు\\{} Urdu(3): میں کانچ کھا سکتا ہوں اور مجھے تکلیف نہیں ہوتی ۔\\{} Pashto(3): زه شيشه خوړلې شم، هغه ما نه خوږوي\\{} Farsi / Persian(3): .من می توانم بدونِ احساس درد شيشه بخورم\\{} Arabic(3): أنا قادر على أكل الزجاج و هذا لا يؤلمني.\\{} Aramaic: (NEEDED)\\{} Maltese: Nista' niekol il{-}ħġieġ u ma jagħmilli xejn.\\{} Hebrew(3): אני יכול לאכול זכוכית וזה לא מזיק לי.\\{} Yiddish(3): איך קען עסן גלאָז און עס טוט מיר נישט װײ.\\{} Judeo{-}Arabic: (NEEDED)\\{} Ladino: (NEEDED)\\{} Gǝʼǝz: (NEEDED)\\{} Amharic: (NEEDED)\\{} Twi: Metumi awe tumpan, ɜnyɜ me hwee.\\{} Hausa (Latin): Inā iya taunar gilāshi kuma in gamā lāfiyā.\\{} Hausa (Ajami) (2): إِنا إِىَ تَونَر غِلَاشِ كُمَ إِن غَمَا لَافِىَا\\{} Yoruba(4): Mo lè je̩ dígí, kò ní pa mí lára.\\{} Lingala: Nakokí kolíya biténi bya milungi, ekosála ngáí mabé tɛ́.\\{} (Ki)Swahili: Naweza kula bilauri na sikunyui.\\{} Malay: Saya boleh makan kaca dan ia tidak mencederakan saya.\\{} Tagalog: Kaya kong kumain nang bubog at hindi ako masaktan.\\{} Chamorro: Siña yo' chumocho krestat, ti ha na'lalamen yo'.\\{} Javanese: Aku isa mangan beling tanpa lara.\\{} Vietnamese (quốc ngữ): Tôi có thể ăn thủy tinh mà không hại gì.\\{} Vietnamese (nôm) (4): 些 𣎏 世 咹 水 晶 𦓡 空 𣎏 害 咦\\{} Thai: ฉันกินกระจกได้ แต่มันไม่ทำให้ฉันเจ็บ\\{} Mongolian (Cyrillic): Би шил идэй чадна, надад хортой биш\\{} Chinese: 我能吞下玻璃而不伤身体。\\{} Chinese (Traditional): 我能吞下玻璃而不傷身體。\\{} Taiwanese(6): Góa ē{-}tàng chia̍h po{-}lê, mā bē tio̍h{-}siong.\\{} Japanese: 私はガラスを食べられます。それは私を傷つけません。\\{} Korean: 나는 유리를 먹을 수 있어요. 그래도 아프지 않아요\\{} Bislama: Mi save kakae glas, hemi no save katem mi.\\{} Hawaiian: Hiki iaʻu ke ʻai i ke aniani; ʻaʻole nō lā au e ʻeha.\\{} Marquesan: E koʻana e kai i te karahi, mea ʻā, ʻaʻe hauhau.\\{} Chinook Jargon: Naika məkmək kakshət labutay, pi weyk ukuk munk{-}sik nay.\\{} Navajo: Tsésǫʼ yishą́ągo bííníshghah dóó doo shił neezgai da.\\{} Cherokee (and Cree, Chickasaw, Cree, Micmac, Ojibwa, Lakota, Inuktitut, Náhuatl, Quechua, Aymara, and other American languages): (NEEDED)\\{} Garifuna: (NEEDED)\\{} Gullah: (NEEDED)\\{} Lojban: mi kakne le nu citka le blaci .iku'i le se go'i na xrani mi\\{} Nórdicg: Ljœr ye caudran créneþ ý jor cẃran.\\{}
\end{document}
