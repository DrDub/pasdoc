\documentclass{report}
\usepackage{hyperref}
% WARNING: THIS SHOULD BE MODIFIED DEPENDING ON THE LETTER/A4 SIZE
\oddsidemargin 0cm
\evensidemargin 0cm
\marginparsep 0cm
\marginparwidth 0cm
\parindent 0cm
\textwidth 16.5cm

\ifpdf
  \usepackage[pdftex]{graphicx}
\else
  \usepackage[dvips]{graphicx}
\fi

% definitons for warning and note tag
\usepackage[most]{tcolorbox}
\newtcolorbox{tcbwarning}{
 breakable,
 enhanced jigsaw,
 top=0pt,
 bottom=0pt,
 titlerule=0pt,
 bottomtitle=0pt,
 rightrule=0pt,
 toprule=0pt,
 bottomrule=0pt,
 colback=white,
 arc=0pt,
 outer arc=0pt,
 title style={white},
 fonttitle=\color{black}\bfseries,
 left=8pt,
 colframe=red,
 title={Warning:},
}
\newtcolorbox{tcbnote}{
 breakable,
 enhanced jigsaw,
 top=0pt,
 bottom=0pt,
 titlerule=0pt,
 bottomtitle=0pt,
 rightrule=0pt,
 toprule=0pt,
 bottomrule=0pt,
 colback=white,
 arc=0pt,
 outer arc=0pt,
 title style={white},
 fonttitle=\color{black}\bfseries,
 left=8pt,
 colframe=yellow,
 title={Note:},
}

\begin{document}
% special variable used for calculating some widths.
\newlength{\tmplength}
\chapter{Unit ok{\_}dashes}
\section{Description}
Test of various dashes.\hfill\vspace*{1ex}



Triple dash produces em{-}dash, for separating parts of sentence and such, like "I know a secret --- but I won't tell".

Double dash produces en{-}dash, intended to use for numbers ranges, like "10--20".

Normal single dash is a short dash, for compound words, like "variable{-}width font".

You can write @{-} in cases where you really want to write just 2 or more consecutive short dashes. E.g. {-}{-}long{-}option{-}name (here I escaped only the 1st "{-}", this means that the rest of dashes is also treated as a short dash), or {-}{-}long{-}option{-}name (here I escaped only the 2nd dash), or {-}{-}long{-}option{-}name (here I escaped two first dashes, which wasn't really necessary, it's sufficient to escape either 1st or the 2nd dash), {-}{-}long{-}option{-}name (here I escaped all dashes; this looks unnecessary ugly in source code, but it's correct).
\end{document}
