\documentclass{report}
\usepackage{hyperref}
% WARNING: THIS SHOULD BE MODIFIED DEPENDING ON THE LETTER/A4 SIZE
\oddsidemargin 0cm
\evensidemargin 0cm
\marginparsep 0cm
\marginparwidth 0cm
\parindent 0cm
\textwidth 16.5cm

\ifpdf
  \usepackage[pdftex]{graphicx}
\else
  \usepackage[dvips]{graphicx}
\fi

\begin{document}
% special variable used for calculating some widths.
\newlength{\tmplength}
\chapter{Unit ok{\_}if{\_}directive}
\section{Description}
PasDoc fails to parse this, it seems that {\$}if is not understood at all.\hfill\vspace*{1ex}



Bug spotted by Michalis on 2005{-}12{-}04 when trying `make htmldocs' on fpc compiler sources, in file globals.pas.

Update 2005{-}12{-}05: {\$}if is sorta handled now, and so is {\$}ifend. This testcase is parsed now. There is still work remaining to be done with regards to {\$}if and {\$}elseif, this is documented on [https://github.com/pasdoc/pasdoc/wiki/ToDo].
\section{Constants}
\subsection*{MathQNaN}
\begin{list}{}{
\settowidth{\tmplength}{\textbf{Description}}
\setlength{\itemindent}{0cm}
\setlength{\listparindent}{0cm}
\setlength{\leftmargin}{\evensidemargin}
\addtolength{\leftmargin}{\tmplength}
\settowidth{\labelsep}{X}
\addtolength{\leftmargin}{\labelsep}
\setlength{\labelwidth}{\tmplength}
}
\begin{flushleft}
\item[\textbf{Declaration}\hfill]
\begin{ttfamily}
MathQNaN : tdoublearray = (0,0,252,255,0,0,0,0);\end{ttfamily}


\end{flushleft}
\end{list}
\subsection*{MathInf}
\begin{list}{}{
\settowidth{\tmplength}{\textbf{Description}}
\setlength{\itemindent}{0cm}
\setlength{\listparindent}{0cm}
\setlength{\leftmargin}{\evensidemargin}
\addtolength{\leftmargin}{\tmplength}
\settowidth{\labelsep}{X}
\addtolength{\leftmargin}{\labelsep}
\setlength{\labelwidth}{\tmplength}
}
\begin{flushleft}
\item[\textbf{Declaration}\hfill]
\begin{ttfamily}
MathInf : tdoublearray = (0,0,240,127,0,0,0,0);\end{ttfamily}


\end{flushleft}
\end{list}
\subsection*{MathNegInf}
\begin{list}{}{
\settowidth{\tmplength}{\textbf{Description}}
\setlength{\itemindent}{0cm}
\setlength{\listparindent}{0cm}
\setlength{\leftmargin}{\evensidemargin}
\addtolength{\leftmargin}{\tmplength}
\settowidth{\labelsep}{X}
\addtolength{\leftmargin}{\labelsep}
\setlength{\labelwidth}{\tmplength}
}
\begin{flushleft}
\item[\textbf{Declaration}\hfill]
\begin{ttfamily}
MathNegInf : tdoublearray = (0,0,240,255,0,0,0,0);\end{ttfamily}


\end{flushleft}
\end{list}
\subsection*{MathPi}
\begin{list}{}{
\settowidth{\tmplength}{\textbf{Description}}
\setlength{\itemindent}{0cm}
\setlength{\listparindent}{0cm}
\setlength{\leftmargin}{\evensidemargin}
\addtolength{\leftmargin}{\tmplength}
\settowidth{\labelsep}{X}
\addtolength{\leftmargin}{\labelsep}
\setlength{\labelwidth}{\tmplength}
}
\begin{flushleft}
\item[\textbf{Declaration}\hfill]
\begin{ttfamily}
MathPi : tdoublearray =  (251,33,9,64,24,45,68,84);\end{ttfamily}


\end{flushleft}
\end{list}
\subsection*{Included}
\begin{list}{}{
\settowidth{\tmplength}{\textbf{Description}}
\setlength{\itemindent}{0cm}
\setlength{\listparindent}{0cm}
\setlength{\leftmargin}{\evensidemargin}
\addtolength{\leftmargin}{\tmplength}
\settowidth{\labelsep}{X}
\addtolength{\leftmargin}{\labelsep}
\setlength{\labelwidth}{\tmplength}
}
\begin{flushleft}
\item[\textbf{Declaration}\hfill]
\begin{ttfamily}
Included = 1;\end{ttfamily}


\end{flushleft}
\end{list}
\end{document}
