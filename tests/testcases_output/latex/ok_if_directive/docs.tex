\documentclass{report}
\usepackage{hyperref}
% WARNING: THIS SHOULD BE MODIFIED DEPENDING ON THE LETTER/A4 SIZE
\oddsidemargin 0cm
\evensidemargin 0cm
\marginparsep 0cm
\marginparwidth 0cm
\parindent 0cm
\setlength{\textwidth}{\paperwidth}
\addtolength{\textwidth}{-2in}


% Conditional define to determine if pdf output is used
\newif\ifpdf
\ifx\pdfoutput\undefined
\pdffalse
\else
\pdfoutput=1
\pdftrue
\fi

\ifpdf
  \usepackage[pdftex]{graphicx}
\else
  \usepackage[dvips]{graphicx}
\fi

% Write Document information for pdflatex/pdftex
\ifpdf
\pdfinfo{
 /Author     (Pasdoc)
 /Title      ()
}
\fi


% definitons for warning and note tag
\usepackage[most]{tcolorbox}
\newtcolorbox{tcbwarning}{
 breakable,
 enhanced jigsaw,
 top=0pt,
 bottom=0pt,
 titlerule=0pt,
 bottomtitle=0pt,
 rightrule=0pt,
 toprule=0pt,
 bottomrule=0pt,
 colback=white,
 arc=0pt,
 outer arc=0pt,
 title style={white},
 fonttitle=\color{black}\bfseries,
 left=8pt,
 colframe=red,
 title={Warning:},
}
\newtcolorbox{tcbnote}{
 breakable,
 enhanced jigsaw,
 top=0pt,
 bottom=0pt,
 titlerule=0pt,
 bottomtitle=0pt,
 rightrule=0pt,
 toprule=0pt,
 bottomrule=0pt,
 colback=white,
 arc=0pt,
 outer arc=0pt,
 title style={white},
 fonttitle=\color{black}\bfseries,
 left=8pt,
 colframe=yellow,
 title={Note:},
}

\begin{document}
\label{toc}\tableofcontents
\newpage
% special variable used for calculating some widths.
\newlength{\tmplength}
\chapter{Unit ok{\_}if{\_}directive}
\label{ok_if_directive}
\index{ok{\_}if{\_}directive}
\section{Constants}
\ifpdf
\subsection*{\large{\textbf{MathQNaN}}\normalsize\hspace{1ex}\hrulefill}
\else
\subsection*{MathQNaN}
\fi
\label{ok_if_directive-MathQNaN}
\index{MathQNaN}
\begin{list}{}{
\settowidth{\tmplength}{\textbf{Description}}
\setlength{\itemindent}{0cm}
\setlength{\listparindent}{0cm}
\setlength{\leftmargin}{\evensidemargin}
\addtolength{\leftmargin}{\tmplength}
\settowidth{\labelsep}{X}
\addtolength{\leftmargin}{\labelsep}
\setlength{\labelwidth}{\tmplength}
}
\item[\textbf{Declaration}\hfill]
\ifpdf
\begin{flushleft}
\fi
\begin{ttfamily}
MathQNaN : tdoublearray = (255,252,0,0,0,0,0,0);\end{ttfamily}

\ifpdf
\end{flushleft}
\fi

\end{list}
\ifpdf
\subsection*{\large{\textbf{MathInf}}\normalsize\hspace{1ex}\hrulefill}
\else
\subsection*{MathInf}
\fi
\label{ok_if_directive-MathInf}
\index{MathInf}
\begin{list}{}{
\settowidth{\tmplength}{\textbf{Description}}
\setlength{\itemindent}{0cm}
\setlength{\listparindent}{0cm}
\setlength{\leftmargin}{\evensidemargin}
\addtolength{\leftmargin}{\tmplength}
\settowidth{\labelsep}{X}
\addtolength{\leftmargin}{\labelsep}
\setlength{\labelwidth}{\tmplength}
}
\item[\textbf{Declaration}\hfill]
\ifpdf
\begin{flushleft}
\fi
\begin{ttfamily}
MathInf : tdoublearray = (127,240,0,0,0,0,0,0);\end{ttfamily}

\ifpdf
\end{flushleft}
\fi

\end{list}
\ifpdf
\subsection*{\large{\textbf{MathNegInf}}\normalsize\hspace{1ex}\hrulefill}
\else
\subsection*{MathNegInf}
\fi
\label{ok_if_directive-MathNegInf}
\index{MathNegInf}
\begin{list}{}{
\settowidth{\tmplength}{\textbf{Description}}
\setlength{\itemindent}{0cm}
\setlength{\listparindent}{0cm}
\setlength{\leftmargin}{\evensidemargin}
\addtolength{\leftmargin}{\tmplength}
\settowidth{\labelsep}{X}
\addtolength{\leftmargin}{\labelsep}
\setlength{\labelwidth}{\tmplength}
}
\item[\textbf{Declaration}\hfill]
\ifpdf
\begin{flushleft}
\fi
\begin{ttfamily}
MathNegInf : tdoublearray = (255,240,0,0,0,0,0,0);\end{ttfamily}

\ifpdf
\end{flushleft}
\fi

\end{list}
\ifpdf
\subsection*{\large{\textbf{MathPi}}\normalsize\hspace{1ex}\hrulefill}
\else
\subsection*{MathPi}
\fi
\label{ok_if_directive-MathPi}
\index{MathPi}
\begin{list}{}{
\settowidth{\tmplength}{\textbf{Description}}
\setlength{\itemindent}{0cm}
\setlength{\listparindent}{0cm}
\setlength{\leftmargin}{\evensidemargin}
\addtolength{\leftmargin}{\tmplength}
\settowidth{\labelsep}{X}
\addtolength{\leftmargin}{\labelsep}
\setlength{\labelwidth}{\tmplength}
}
\item[\textbf{Declaration}\hfill]
\ifpdf
\begin{flushleft}
\fi
\begin{ttfamily}
MathPi : tdoublearray =  (64,9,33,251,84,68,45,24);\end{ttfamily}

\ifpdf
\end{flushleft}
\fi

\end{list}
\ifpdf
\subsection*{\large{\textbf{MathPiExtended}}\normalsize\hspace{1ex}\hrulefill}
\else
\subsection*{MathPiExtended}
\fi
\label{ok_if_directive-MathPiExtended}
\index{MathPiExtended}
\begin{list}{}{
\settowidth{\tmplength}{\textbf{Description}}
\setlength{\itemindent}{0cm}
\setlength{\listparindent}{0cm}
\setlength{\leftmargin}{\evensidemargin}
\addtolength{\leftmargin}{\tmplength}
\settowidth{\labelsep}{X}
\addtolength{\leftmargin}{\labelsep}
\setlength{\labelwidth}{\tmplength}
}
\item[\textbf{Declaration}\hfill]
\ifpdf
\begin{flushleft}
\fi
\begin{ttfamily}
MathPiExtended : textendedarray = (64,0,201,15,218,162,33,104,194,53);\end{ttfamily}

\ifpdf
\end{flushleft}
\fi

\end{list}
\ifpdf
\subsection*{\large{\textbf{Foo2}}\normalsize\hspace{1ex}\hrulefill}
\else
\subsection*{Foo2}
\fi
\label{ok_if_directive-Foo2}
\index{Foo2}
\begin{list}{}{
\settowidth{\tmplength}{\textbf{Description}}
\setlength{\itemindent}{0cm}
\setlength{\listparindent}{0cm}
\setlength{\leftmargin}{\evensidemargin}
\addtolength{\leftmargin}{\tmplength}
\settowidth{\labelsep}{X}
\addtolength{\leftmargin}{\labelsep}
\setlength{\labelwidth}{\tmplength}
}
\item[\textbf{Declaration}\hfill]
\ifpdf
\begin{flushleft}
\fi
\begin{ttfamily}
Foo2 = 1;\end{ttfamily}

\ifpdf
\end{flushleft}
\fi

\end{list}
\ifpdf
\subsection*{\large{\textbf{Foo4}}\normalsize\hspace{1ex}\hrulefill}
\else
\subsection*{Foo4}
\fi
\label{ok_if_directive-Foo4}
\index{Foo4}
\begin{list}{}{
\settowidth{\tmplength}{\textbf{Description}}
\setlength{\itemindent}{0cm}
\setlength{\listparindent}{0cm}
\setlength{\leftmargin}{\evensidemargin}
\addtolength{\leftmargin}{\tmplength}
\settowidth{\labelsep}{X}
\addtolength{\leftmargin}{\labelsep}
\setlength{\labelwidth}{\tmplength}
}
\item[\textbf{Declaration}\hfill]
\ifpdf
\begin{flushleft}
\fi
\begin{ttfamily}
Foo4 = 1;\end{ttfamily}

\ifpdf
\end{flushleft}
\fi

\end{list}
\end{document}
