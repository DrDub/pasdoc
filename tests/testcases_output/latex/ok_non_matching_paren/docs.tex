\documentclass{report}
\usepackage{hyperref}
% WARNING: THIS SHOULD BE MODIFIED DEPENDING ON THE LETTER/A4 SIZE
\oddsidemargin 0cm
\evensidemargin 0cm
\marginparsep 0cm
\marginparwidth 0cm
\parindent 0cm
\setlength{\textwidth}{\paperwidth}
\addtolength{\textwidth}{-2in}


% Conditional define to determine if pdf output is used
\newif\ifpdf
\ifx\pdfoutput\undefined
\pdffalse
\else
\pdfoutput=1
\pdftrue
\fi

\ifpdf
  \usepackage[pdftex]{graphicx}
\else
  \usepackage[dvips]{graphicx}
\fi

% Write Document information for pdflatex/pdftex
\ifpdf
\pdfinfo{
 /Author     (Pasdoc)
 /Title      ()
}
\fi


\begin{document}
\label{toc}\tableofcontents
\newpage
% special variable used for calculating some widths.
\newlength{\tmplength}
\chapter{Unit ok{\_}non{\_}matching{\_}paren}
\label{ok_non_matching_paren}
\index{ok{\_}non{\_}matching{\_}paren}
\section{Description}
Test of @( and @) constructs.\hfill\vspace*{1ex}



\textbf{This is bold, followed by two "at" chars and two parens. @@( ) } No longer bold.

\textit{This is italic, followed by one "at" char and one opening paren. @( } No longer italic.

\textbf{This is bold, followed by two parens. ( ) } No longer bold.

\textit{This is italic, followed by one closing paren. ) } No longer italic.

\textit{This is bold, followed by "at" char. @} No longer italic.
\end{document}
