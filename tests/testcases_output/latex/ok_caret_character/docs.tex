\documentclass{report}
\usepackage{hyperref}
% WARNING: THIS SHOULD BE MODIFIED DEPENDING ON THE LETTER/A4 SIZE
\oddsidemargin 0cm
\evensidemargin 0cm
\marginparsep 0cm
\marginparwidth 0cm
\parindent 0cm
\setlength{\textwidth}{\paperwidth}
\addtolength{\textwidth}{-2in}


% Conditional define to determine if pdf output is used
\newif\ifpdf
\ifx\pdfoutput\undefined
\pdffalse
\else
\pdfoutput=1
\pdftrue
\fi

\ifpdf
  \usepackage[pdftex]{graphicx}
\else
  \usepackage[dvips]{graphicx}
\fi

% Write Document information for pdflatex/pdftex
\ifpdf
\pdfinfo{
 /Author     (Pasdoc)
 /Title      ()
}
\fi


% definitons for warning and note tag
\usepackage[most]{tcolorbox}
\newtcolorbox{tcbwarning}{
 breakable,
 enhanced jigsaw,
 top=0pt,
 bottom=0pt,
 titlerule=0pt,
 bottomtitle=0pt,
 rightrule=0pt,
 toprule=0pt,
 bottomrule=0pt,
 colback=white,
 arc=0pt,
 outer arc=0pt,
 title style={white},
 fonttitle=\color{black}\bfseries,
 left=8pt,
 colframe=red,
 title={Warning:},
}
\newtcolorbox{tcbnote}{
 breakable,
 enhanced jigsaw,
 top=0pt,
 bottom=0pt,
 titlerule=0pt,
 bottomtitle=0pt,
 rightrule=0pt,
 toprule=0pt,
 bottomrule=0pt,
 colback=white,
 arc=0pt,
 outer arc=0pt,
 title style={white},
 fonttitle=\color{black}\bfseries,
 left=8pt,
 colframe=yellow,
 title={Note:},
}

\begin{document}
\label{toc}\tableofcontents
\newpage
% special variable used for calculating some widths.
\newlength{\tmplength}
\chapter{Unit ok{\_}caret{\_}character}
\label{ok_caret_character}
\index{ok{\_}caret{\_}character}
\section{Description}
Test of parsing "{\^{}}char" constants, see [\href{http://www.freepascal.org/docs-html/ref/refsu7.html}{http://www.freepascal.org/docs-html/ref/refsu7.html}].\hfill\vspace*{1ex}



Note that while FPC docs say that only {\^{}}A .. {\^{}}Z are allowed, actually both FPC 2.0.0 and Kylix 3 accept other characters after the caret, e.g. "{\^{}}{\textbackslash}" is accepted.

See also bug report "[ 1358911 ] Invalid character in input stream".
\section{Constants}
\ifpdf
\subsection*{\large{\textbf{C2}}\normalsize\hspace{1ex}\hrulefill}
\else
\subsection*{C2}
\fi
\label{ok_caret_character-C2}
\index{C2}
\begin{list}{}{
\settowidth{\tmplength}{\textbf{Description}}
\setlength{\itemindent}{0cm}
\setlength{\listparindent}{0cm}
\setlength{\leftmargin}{\evensidemargin}
\addtolength{\leftmargin}{\tmplength}
\settowidth{\labelsep}{X}
\addtolength{\leftmargin}{\labelsep}
\setlength{\labelwidth}{\tmplength}
}
\item[\textbf{Declaration}\hfill]
\ifpdf
\begin{flushleft}
\fi
\begin{ttfamily}
C2 = {\^{}}A;\end{ttfamily}

\ifpdf
\end{flushleft}
\fi

\end{list}
\ifpdf
\subsection*{\large{\textbf{C3}}\normalsize\hspace{1ex}\hrulefill}
\else
\subsection*{C3}
\fi
\label{ok_caret_character-C3}
\index{C3}
\begin{list}{}{
\settowidth{\tmplength}{\textbf{Description}}
\setlength{\itemindent}{0cm}
\setlength{\listparindent}{0cm}
\setlength{\leftmargin}{\evensidemargin}
\addtolength{\leftmargin}{\tmplength}
\settowidth{\labelsep}{X}
\addtolength{\leftmargin}{\labelsep}
\setlength{\labelwidth}{\tmplength}
}
\item[\textbf{Declaration}\hfill]
\ifpdf
\begin{flushleft}
\fi
\begin{ttfamily}
C3 = {\^{}}B;\end{ttfamily}

\ifpdf
\end{flushleft}
\fi

\end{list}
\ifpdf
\subsection*{\large{\textbf{C4}}\normalsize\hspace{1ex}\hrulefill}
\else
\subsection*{C4}
\fi
\label{ok_caret_character-C4}
\index{C4}
\begin{list}{}{
\settowidth{\tmplength}{\textbf{Description}}
\setlength{\itemindent}{0cm}
\setlength{\listparindent}{0cm}
\setlength{\leftmargin}{\evensidemargin}
\addtolength{\leftmargin}{\tmplength}
\settowidth{\labelsep}{X}
\addtolength{\leftmargin}{\labelsep}
\setlength{\labelwidth}{\tmplength}
}
\item[\textbf{Declaration}\hfill]
\ifpdf
\begin{flushleft}
\fi
\begin{ttfamily}
C4 = {\^{}}Z;\end{ttfamily}

\ifpdf
\end{flushleft}
\fi

\end{list}
\ifpdf
\subsection*{\large{\textbf{SUB}}\normalsize\hspace{1ex}\hrulefill}
\else
\subsection*{SUB}
\fi
\label{ok_caret_character-SUB}
\index{SUB}
\begin{list}{}{
\settowidth{\tmplength}{\textbf{Description}}
\setlength{\itemindent}{0cm}
\setlength{\listparindent}{0cm}
\setlength{\leftmargin}{\evensidemargin}
\addtolength{\leftmargin}{\tmplength}
\settowidth{\labelsep}{X}
\addtolength{\leftmargin}{\labelsep}
\setlength{\labelwidth}{\tmplength}
}
\item[\textbf{Declaration}\hfill]
\ifpdf
\begin{flushleft}
\fi
\begin{ttfamily}
SUB = {\^{}}Z;\end{ttfamily}

\ifpdf
\end{flushleft}
\fi

\par
\item[\textbf{Description}]
SUBstitute

\end{list}
\ifpdf
\subsection*{\large{\textbf{ESC}}\normalsize\hspace{1ex}\hrulefill}
\else
\subsection*{ESC}
\fi
\label{ok_caret_character-ESC}
\index{ESC}
\begin{list}{}{
\settowidth{\tmplength}{\textbf{Description}}
\setlength{\itemindent}{0cm}
\setlength{\listparindent}{0cm}
\setlength{\leftmargin}{\evensidemargin}
\addtolength{\leftmargin}{\tmplength}
\settowidth{\labelsep}{X}
\addtolength{\leftmargin}{\labelsep}
\setlength{\labelwidth}{\tmplength}
}
\item[\textbf{Declaration}\hfill]
\ifpdf
\begin{flushleft}
\fi
\begin{ttfamily}
ESC = {\^{}}[;\end{ttfamily}

\ifpdf
\end{flushleft}
\fi

\par
\item[\textbf{Description}]
ESCape

\end{list}
\ifpdf
\subsection*{\large{\textbf{FS}}\normalsize\hspace{1ex}\hrulefill}
\else
\subsection*{FS}
\fi
\label{ok_caret_character-FS}
\index{FS}
\begin{list}{}{
\settowidth{\tmplength}{\textbf{Description}}
\setlength{\itemindent}{0cm}
\setlength{\listparindent}{0cm}
\setlength{\leftmargin}{\evensidemargin}
\addtolength{\leftmargin}{\tmplength}
\settowidth{\labelsep}{X}
\addtolength{\leftmargin}{\labelsep}
\setlength{\labelwidth}{\tmplength}
}
\item[\textbf{Declaration}\hfill]
\ifpdf
\begin{flushleft}
\fi
\begin{ttfamily}
FS  = {\^{}}{\textbackslash};\end{ttfamily}

\ifpdf
\end{flushleft}
\fi

\par
\item[\textbf{Description}]
File Separator

\end{list}
\ifpdf
\subsection*{\large{\textbf{GS}}\normalsize\hspace{1ex}\hrulefill}
\else
\subsection*{GS}
\fi
\label{ok_caret_character-GS}
\index{GS}
\begin{list}{}{
\settowidth{\tmplength}{\textbf{Description}}
\setlength{\itemindent}{0cm}
\setlength{\listparindent}{0cm}
\setlength{\leftmargin}{\evensidemargin}
\addtolength{\leftmargin}{\tmplength}
\settowidth{\labelsep}{X}
\addtolength{\leftmargin}{\labelsep}
\setlength{\labelwidth}{\tmplength}
}
\item[\textbf{Declaration}\hfill]
\ifpdf
\begin{flushleft}
\fi
\begin{ttfamily}
GS  = {\^{}}];\end{ttfamily}

\ifpdf
\end{flushleft}
\fi

\par
\item[\textbf{Description}]
Group Separator

\end{list}
\ifpdf
\subsection*{\large{\textbf{RS}}\normalsize\hspace{1ex}\hrulefill}
\else
\subsection*{RS}
\fi
\label{ok_caret_character-RS}
\index{RS}
\begin{list}{}{
\settowidth{\tmplength}{\textbf{Description}}
\setlength{\itemindent}{0cm}
\setlength{\listparindent}{0cm}
\setlength{\leftmargin}{\evensidemargin}
\addtolength{\leftmargin}{\tmplength}
\settowidth{\labelsep}{X}
\addtolength{\leftmargin}{\labelsep}
\setlength{\labelwidth}{\tmplength}
}
\item[\textbf{Declaration}\hfill]
\ifpdf
\begin{flushleft}
\fi
\begin{ttfamily}
RS  = {\^{}}{\^{}};\end{ttfamily}

\ifpdf
\end{flushleft}
\fi

\par
\item[\textbf{Description}]
Record Separator

\end{list}
\ifpdf
\subsection*{\large{\textbf{US}}\normalsize\hspace{1ex}\hrulefill}
\else
\subsection*{US}
\fi
\label{ok_caret_character-US}
\index{US}
\begin{list}{}{
\settowidth{\tmplength}{\textbf{Description}}
\setlength{\itemindent}{0cm}
\setlength{\listparindent}{0cm}
\setlength{\leftmargin}{\evensidemargin}
\addtolength{\leftmargin}{\tmplength}
\settowidth{\labelsep}{X}
\addtolength{\leftmargin}{\labelsep}
\setlength{\labelwidth}{\tmplength}
}
\item[\textbf{Declaration}\hfill]
\ifpdf
\begin{flushleft}
\fi
\begin{ttfamily}
US  = {\^{}}{\_};\end{ttfamily}

\ifpdf
\end{flushleft}
\fi

\par
\item[\textbf{Description}]
Unit Separator

\end{list}
\end{document}
