\documentclass{report}
\usepackage{hyperref}
% WARNING: THIS SHOULD BE MODIFIED DEPENDING ON THE LETTER/A4 SIZE
\oddsidemargin 0cm
\evensidemargin 0cm
\marginparsep 0cm
\marginparwidth 0cm
\parindent 0cm
\setlength{\textwidth}{\paperwidth}
\addtolength{\textwidth}{-2in}


% Conditional define to determine if pdf output is used
\newif\ifpdf
\ifx\pdfoutput\undefined
\pdffalse
\else
\pdfoutput=1
\pdftrue
\fi

\ifpdf
  \usepackage[pdftex]{graphicx}
\else
  \usepackage[dvips]{graphicx}
\fi

% Write Document information for pdflatex/pdftex
\ifpdf
\pdfinfo{
 /Author     (Pasdoc)
 /Title      ()
}
\fi


\begin{document}
\label{toc}\tableofcontents
\newpage
% special variable used for calculating some widths.
\newlength{\tmplength}
\chapter{Unit ok{\_}link{\_}1{\_}char}
\label{ok_link_1_char}
\index{ok{\_}link{\_}1{\_}char}
\section{Overview}
\begin{description}
\item[\texttt{\begin{ttfamily}TMeasurementPointItem\end{ttfamily} Class}]\begin{ttfamily}TMeasurementPointItem\end{ttfamily} is used to specify the real{-}world coordinates of a pixel in a bitmap.
\end{description}
\section{Classes, Interfaces, Objects and Records}
\ifpdf
\subsection*{\large{\textbf{TMeasurementPointItem Class}}\normalsize\hspace{1ex}\hrulefill}
\else
\subsection*{TMeasurementPointItem Class}
\fi
\label{ok_link_1_char.TMeasurementPointItem}
\index{TMeasurementPointItem}
\subsubsection*{\large{\textbf{Hierarchy}}\normalsize\hspace{1ex}\hfill}
TMeasurementPointItem {$>$} TCollectionItem
\subsubsection*{\large{\textbf{Description}}\normalsize\hspace{1ex}\hfill}
\begin{ttfamily}TMeasurementPointItem\end{ttfamily} is used to specify the real{-}world coordinates of a pixel in a bitmap.\subsubsection*{\large{\textbf{Properties}}\normalsize\hspace{1ex}\hfill}
\begin{list}{}{
\settowidth{\tmplength}{\textbf{X}}
\setlength{\itemindent}{0cm}
\setlength{\listparindent}{0cm}
\setlength{\leftmargin}{\evensidemargin}
\addtolength{\leftmargin}{\tmplength}
\settowidth{\labelsep}{X}
\addtolength{\leftmargin}{\labelsep}
\setlength{\labelwidth}{\tmplength}
}
\label{ok_link_1_char.TMeasurementPointItem-X}
\index{X}
\item[\textbf{X}\hfill]
\ifpdf
\begin{flushleft}
\fi
\begin{ttfamily}
published property X: double read FX write SetX;\end{ttfamily}

\ifpdf
\end{flushleft}
\fi


\par \begin{ttfamily}X\end{ttfamily} is the X real{-}world coordinate.\label{ok_link_1_char.TMeasurementPointItem-Y}
\index{Y}
\item[\textbf{Y}\hfill]
\ifpdf
\begin{flushleft}
\fi
\begin{ttfamily}
published property Y: double read FY write SetY;\end{ttfamily}

\ifpdf
\end{flushleft}
\fi


\par \begin{ttfamily}Y\end{ttfamily} is the Y real{-}world coordinate.\end{list}
\subsubsection*{\large{\textbf{Fields}}\normalsize\hspace{1ex}\hfill}
\begin{list}{}{
\settowidth{\tmplength}{\textbf{FX}}
\setlength{\itemindent}{0cm}
\setlength{\listparindent}{0cm}
\setlength{\leftmargin}{\evensidemargin}
\addtolength{\leftmargin}{\tmplength}
\settowidth{\labelsep}{X}
\addtolength{\leftmargin}{\labelsep}
\setlength{\labelwidth}{\tmplength}
}
\label{ok_link_1_char.TMeasurementPointItem-FX}
\index{FX}
\item[\textbf{FX}\hfill]
\ifpdf
\begin{flushleft}
\fi
\begin{ttfamily}
private FX: double;\end{ttfamily}

\ifpdf
\end{flushleft}
\fi


\par \begin{ttfamily}FX\end{ttfamily}: double; See \begin{ttfamily}X\end{ttfamily}(\ref{ok_link_1_char.TMeasurementPointItem-X}).\label{ok_link_1_char.TMeasurementPointItem-FY}
\index{FY}
\item[\textbf{FY}\hfill]
\ifpdf
\begin{flushleft}
\fi
\begin{ttfamily}
private FY: double;\end{ttfamily}

\ifpdf
\end{flushleft}
\fi


\par \begin{ttfamily}FY\end{ttfamily}: double; See \begin{ttfamily}Y\end{ttfamily}(\ref{ok_link_1_char.TMeasurementPointItem-Y}).\end{list}
\subsubsection*{\large{\textbf{Methods}}\normalsize\hspace{1ex}\hfill}
\paragraph*{SetX}\hspace*{\fill}

\label{ok_link_1_char.TMeasurementPointItem-SetX}
\index{SetX}
\begin{list}{}{
\settowidth{\tmplength}{\textbf{Description}}
\setlength{\itemindent}{0cm}
\setlength{\listparindent}{0cm}
\setlength{\leftmargin}{\evensidemargin}
\addtolength{\leftmargin}{\tmplength}
\settowidth{\labelsep}{X}
\addtolength{\leftmargin}{\labelsep}
\setlength{\labelwidth}{\tmplength}
}
\item[\textbf{Declaration}\hfill]
\ifpdf
\begin{flushleft}
\fi
\begin{ttfamily}
private procedure SetX(const Value: double);\end{ttfamily}

\ifpdf
\end{flushleft}
\fi

\par
\item[\textbf{Description}]
See \begin{ttfamily}X\end{ttfamily}(\ref{ok_link_1_char.TMeasurementPointItem-X}).

\end{list}
\paragraph*{SetY}\hspace*{\fill}

\label{ok_link_1_char.TMeasurementPointItem-SetY}
\index{SetY}
\begin{list}{}{
\settowidth{\tmplength}{\textbf{Description}}
\setlength{\itemindent}{0cm}
\setlength{\listparindent}{0cm}
\setlength{\leftmargin}{\evensidemargin}
\addtolength{\leftmargin}{\tmplength}
\settowidth{\labelsep}{X}
\addtolength{\leftmargin}{\labelsep}
\setlength{\labelwidth}{\tmplength}
}
\item[\textbf{Declaration}\hfill]
\ifpdf
\begin{flushleft}
\fi
\begin{ttfamily}
private procedure SetY(const Value: double);\end{ttfamily}

\ifpdf
\end{flushleft}
\fi

\par
\item[\textbf{Description}]
See \begin{ttfamily}Y\end{ttfamily}(\ref{ok_link_1_char.TMeasurementPointItem-Y}).

\end{list}
\end{document}
