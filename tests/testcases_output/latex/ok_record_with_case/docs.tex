\documentclass{report}
\usepackage{hyperref}
% WARNING: THIS SHOULD BE MODIFIED DEPENDING ON THE LETTER/A4 SIZE
\oddsidemargin 0cm
\evensidemargin 0cm
\marginparsep 0cm
\marginparwidth 0cm
\parindent 0cm
\setlength{\textwidth}{\paperwidth}
\addtolength{\textwidth}{-2in}


% Conditional define to determine if pdf output is used
\newif\ifpdf
\ifx\pdfoutput\undefined
\pdffalse
\else
\pdfoutput=1
\pdftrue
\fi

\ifpdf
  \usepackage[pdftex]{graphicx}
\else
  \usepackage[dvips]{graphicx}
\fi

% Write Document information for pdflatex/pdftex
\ifpdf
\pdfinfo{
 /Author     (Pasdoc)
 /Title      ()
}
\fi


\begin{document}
\label{toc}\tableofcontents
\newpage
% special variable used for calculating some widths.
\newlength{\tmplength}
\chapter{Unit ok{\_}record{\_}with{\_}case}
\label{ok_record_with_case}
\index{ok{\_}record{\_}with{\_}case}
\section{Description}
Testing of parsing and making docs for records with case parts.\hfill\vspace*{1ex}



It does not pass properly yet: \begin{enumerate}
\setcounter{enumi}{0} \setcounter{enumii}{0} \setcounter{enumiii}{0} \setcounter{enumiv}{0} 
\item Types for fields in record case are not printed (because parser does not set their FullDeclaration properties).
\setcounter{enumi}{1} \setcounter{enumii}{1} \setcounter{enumiii}{1} \setcounter{enumiv}{1} 
\item Also, CaseTwoB field has no description (but it should have "Description of CaseTwoA and CaseTwoB")
\end{enumerate}

Update 2005{-}10{-}17: now this test passes OK, both problems above are solved.
\section{Overview}
\begin{description}
\item[\texttt{\begin{ttfamily}TMyRecord1\end{ttfamily} Record}]
\item[\texttt{\begin{ttfamily}TMyRecord2\end{ttfamily} Record}]
\end{description}
\section{Classes, Interfaces, Objects and Records}
\ifpdf
\subsection*{\large{\textbf{TMyRecord1 Record}}\normalsize\hspace{1ex}\hrulefill}
\else
\subsection*{TMyRecord1 Record}
\fi
\label{ok_record_with_case.TMyRecord1}
\index{TMyRecord1}
%%%%Description
\subsubsection*{\large{\textbf{Fields}}\normalsize\hspace{1ex}\hfill}
\begin{list}{}{
\settowidth{\tmplength}{\textbf{CaseOneSingle}}
\setlength{\itemindent}{0cm}
\setlength{\listparindent}{0cm}
\setlength{\leftmargin}{\evensidemargin}
\addtolength{\leftmargin}{\tmplength}
\settowidth{\labelsep}{X}
\addtolength{\leftmargin}{\labelsep}
\setlength{\labelwidth}{\tmplength}
}
\label{ok_record_with_case.TMyRecord1-NormalField}
\index{NormalField}
\item[\textbf{NormalField}\hfill]
\ifpdf
\begin{flushleft}
\fi
\begin{ttfamily}
public NormalField: Integer;\end{ttfamily}

\ifpdf
\end{flushleft}
\fi


\par Description of NormalField\label{ok_record_with_case.TMyRecord1-CaseDecision}
\index{CaseDecision}
\item[\textbf{CaseDecision}\hfill]
\ifpdf
\begin{flushleft}
\fi
\begin{ttfamily}
published CaseDecision: boolean\end{ttfamily}

\ifpdf
\end{flushleft}
\fi


\par Description of CaseDecision\label{ok_record_with_case.TMyRecord1-CaseOneSingle}
\index{CaseOneSingle}
\item[\textbf{CaseOneSingle}\hfill]
\ifpdf
\begin{flushleft}
\fi
\begin{ttfamily}
public CaseOneSingle: Single\end{ttfamily}

\ifpdf
\end{flushleft}
\fi


\par Description of CaseOneSingle\label{ok_record_with_case.TMyRecord1-CaseTwoSingle}
\index{CaseTwoSingle}
\item[\textbf{CaseTwoSingle}\hfill]
\ifpdf
\begin{flushleft}
\fi
\begin{ttfamily}
public CaseTwoSingle: Single;\end{ttfamily}

\ifpdf
\end{flushleft}
\fi


\par Description of CaseTwoSingle\label{ok_record_with_case.TMyRecord1-CaseTwoInt}
\index{CaseTwoInt}
\item[\textbf{CaseTwoInt}\hfill]
\ifpdf
\begin{flushleft}
\fi
\begin{ttfamily}
public CaseTwoInt: Integer;\end{ttfamily}

\ifpdf
\end{flushleft}
\fi


\par Description of CaseTwoInt\label{ok_record_with_case.TMyRecord1-CaseTwoA}
\index{CaseTwoA}
\item[\textbf{CaseTwoA}\hfill]
\ifpdf
\begin{flushleft}
\fi
\begin{ttfamily}
public CaseTwoA: Integer\end{ttfamily}

\ifpdf
\end{flushleft}
\fi


\par Description of CaseTwoA and CaseTwoB\label{ok_record_with_case.TMyRecord1-CaseTwoB}
\index{CaseTwoB}
\item[\textbf{CaseTwoB}\hfill]
\ifpdf
\begin{flushleft}
\fi
\begin{ttfamily}
public CaseTwoB: Integer\end{ttfamily}

\ifpdf
\end{flushleft}
\fi


\par Description of CaseTwoA and CaseTwoB\end{list}
\ifpdf
\subsection*{\large{\textbf{TMyRecord2 Record}}\normalsize\hspace{1ex}\hrulefill}
\else
\subsection*{TMyRecord2 Record}
\fi
\label{ok_record_with_case.TMyRecord2}
\index{TMyRecord2}
%%%%Description
\subsubsection*{\large{\textbf{Fields}}\normalsize\hspace{1ex}\hfill}
\begin{list}{}{
\settowidth{\tmplength}{\textbf{CaseOneSingle}}
\setlength{\itemindent}{0cm}
\setlength{\listparindent}{0cm}
\setlength{\leftmargin}{\evensidemargin}
\addtolength{\leftmargin}{\tmplength}
\settowidth{\labelsep}{X}
\addtolength{\leftmargin}{\labelsep}
\setlength{\labelwidth}{\tmplength}
}
\label{ok_record_with_case.TMyRecord2-NormalField}
\index{NormalField}
\item[\textbf{NormalField}\hfill]
\ifpdf
\begin{flushleft}
\fi
\begin{ttfamily}
public NormalField: Integer;\end{ttfamily}

\ifpdf
\end{flushleft}
\fi


\par Description of NormalField\label{ok_record_with_case.TMyRecord2-CaseOneSingle}
\index{CaseOneSingle}
\item[\textbf{CaseOneSingle}\hfill]
\ifpdf
\begin{flushleft}
\fi
\begin{ttfamily}
public CaseOneSingle: Single\end{ttfamily}

\ifpdf
\end{flushleft}
\fi


\par Description of CaseOneSingle\label{ok_record_with_case.TMyRecord2-CaseTwoSingle}
\index{CaseTwoSingle}
\item[\textbf{CaseTwoSingle}\hfill]
\ifpdf
\begin{flushleft}
\fi
\begin{ttfamily}
public CaseTwoSingle: Single;\end{ttfamily}

\ifpdf
\end{flushleft}
\fi


\par Description of CaseTwoSingle\label{ok_record_with_case.TMyRecord2-CaseTwoInt}
\index{CaseTwoInt}
\item[\textbf{CaseTwoInt}\hfill]
\ifpdf
\begin{flushleft}
\fi
\begin{ttfamily}
public CaseTwoInt: Integer;\end{ttfamily}

\ifpdf
\end{flushleft}
\fi


\par Description of CaseTwoInt\label{ok_record_with_case.TMyRecord2-CaseTwoA}
\index{CaseTwoA}
\item[\textbf{CaseTwoA}\hfill]
\ifpdf
\begin{flushleft}
\fi
\begin{ttfamily}
public CaseTwoA: Integer\end{ttfamily}

\ifpdf
\end{flushleft}
\fi


\par Description of CaseTwoA and CaseTwoB\label{ok_record_with_case.TMyRecord2-CaseTwoB}
\index{CaseTwoB}
\item[\textbf{CaseTwoB}\hfill]
\ifpdf
\begin{flushleft}
\fi
\begin{ttfamily}
public CaseTwoB: Integer\end{ttfamily}

\ifpdf
\end{flushleft}
\fi


\par Description of CaseTwoA and CaseTwoB\end{list}
\end{document}
